\documentclass[a4paper,12pt]{article}

\usepackage[iso,english]{isodate}
\usepackage[margin=1in]{geometry}
\usepackage[english]{babel}
\usepackage{hyperref}
\usepackage{parskip}

\usepackage[color]{tiscreen}
\usepackage{multicol}
\usepackage{array}
\usepackage{longtable}
\usepackage{fancyvrb-ex}
\fvset{
	frame=single,
	label=\fbox{Source},
	framesep=4mm,
	numbers=left,
	fontsize=\footnotesize,
}

\title{TI calculator screen (and buttons)\\{\small TI-82 STATS}}
\author{Mustafa Ibrahim}

\begin{document}

\maketitle

\begin{center}
	\begin{tabular}{cc}
		\tiscreen
		|10^6*(4^7+1){rarrow}X  |
		|       1.6385E10|
		|{sqrt}(X)*X/10!      |
		|     577971782.1|
		|cos{ar}(cos(Ans))  |
		|        62.11246|
		|{fcur}               |
		|                |
		&
		\tibtnextramatrix
	\end{tabular}

	\dotfill

	\begin{tabular}{ccccc}
		\tibtnsecond & \tibtnmode   & \tibtndel       &                  &             \\
		\tibtnalpha  & \tibtnxton   & \tibtnstat      &                  &             \\
		\tibtnmath   & \tibtnmatrix & \tibtnprgm      & \tibtnvars       & \tibtnclear \\
		\tibtnxnone  & \tibtnsin    & \tibtncos       & \tibtntan        & \tibtnpower \\
		\tibtnxtwo   & \tibtncomma  & \tibtnleftparen & \tibtnrightparen & \tibtndiv   \\
		\tibtnlog    & \tibtnseven  & \tibtneight     & \tibtnnine       & \tibtntimes \\
		\tibtnln     & \tibtnfour   & \tibtnfive      & \tibtnsix        & \tibtnminus \\
		\tibtnsto    & \tibtnone    & \tibtntwo       & \tibtnthree      & \tibtnplus  \\
		\tibtnon     & \tibtnzero   & \tibtndot       & \tibtnneg        & \tibtnenter \\
	\end{tabular}
\end{center}

\newpage\tableofcontents\newpage

\section{Quickstart}

\begin{SideBySideExample}[xrightmargin=5.5cm]
%\usepackage[color]{tiscreen}

\tiscreen
|4+1             |
|               5|
|Ans{sq}         |
|              25|
|{fcur}          |
|                |
|                |
|                |

Lorem \tibtnmath{} ipsum

\tibtnextramath
\end{SideBySideExample}

\section{Package option(s)}

\subsection{Color} \label{sec:color}

Using the \texttt{color} option will change the colors used by the \verb|\LCD|
command for printing the screen using \verb|\tiscreen|. The colors are defined
as \texttt{tiscreenfg} (foreground, i.e. font color) and \texttt{tiscreenbg}
(background) and redefined like this:

\begin{Verbatim}
% Add this to your preamble
\definecolor{tiscreenbg}{HTML}{5d9345}
\definecolor{tiscreenfg}{HTML}{FFFFFF}
\end{Verbatim}

\subsection{Defined colors} \label{sec:colordef}

\begin{tabular}{lcl}
	Name            & Color                                                              & Usage        \\ \hline
	tiscreenfg      & {\ttfamily \color{tiscreenfg}{000000}}                             & LCD commands \\
	tiscreenbg      & {\ttfamily \color{tiscreenbg}{5d9345}}                             & LCD commands \\
	tibtngray       & {\ttfamily \color{tibtngray}{494F54}}                              & Buttons      \\
	tibtnyellow     & {\ttfamily \colorbox{tibtncaseblack}{\color{tibtnyellow}{F1E78B}}} & Buttons      \\
	tibtngreen      & {\ttfamily \colorbox{tibtncaseblack}{\color{tibtngreen}{D5DE93}}}  & Buttons      \\
	tibtnwhite      & {\ttfamily \colorbox{tibtncaseblack}{\color{tibtnwhite}{E2E2E2}}}  & Buttons      \\
	tibtnblue       & {\ttfamily \colorbox{tibtncaseblack}{\color{tibtnblue}{CDE2E9}}}   & Buttons      \\
	tibtncaseblack  & {\ttfamily \color{tibtncaseblack}{232A32}}                         & Buttons      \\
\end{tabular}

\part{Screen}

\section{LCD size}

The default LCD size is $8 \times 16$ (the size of the TI-82 STATS). It can be
changed by redefining the variables used to determine the size of the display
or by using the original \verb|\LCD| command.

\begin{Verbatim}
% First method (For entire document)
\def\tiscreenX{16}
\def\tiscreenY{8}

% Second method (Only once)
\LCD{5}{11}
|ANOTHER  |
|EXAMPLE  |
|WITH A   |
|DIFFERENT|
|SIZE     |
\end{Verbatim}

\section{(Re)defined characters} \LCDcolors{black}{white}

\begin{center}
	\begin{multicols}{2}
		\begin{tabular}{|l|l|l|}
			\multicolumn{3}{c}{\textbf{Added characters}}                    \\ \hline
			Name       & Symbol                         & \verb|\LCD| Code   \\ \hline
			E          & {\large \textLCD{1}|{sciE}~|}  & \texttt{\{sciE\}}  \\
			$\sigma$   & {\large \textLCD{1}|{sigma}~|} & \texttt{\{sigma\}} \\
			$x^2$      & {\large \textLCD{1}|{sq}~|}    & \texttt{\{sq\}}    \\
			$x^{-1}$   & {\large \textLCD{1}|{ar}~|}    & \texttt{\{ar\}}    \\
			$y^x$      & {\large \textLCD{1}|{cx}~|}    & \texttt{\{cx\}}    \\
			$x_1$      & {\large \textLCD{1}|{sub1}~|}  & \texttt{\{sub1\}}  \\
			$x_2$      & {\large \textLCD{1}|{sub2}~|}  & \texttt{\{sub2\}}  \\
			$x_3$      & {\large \textLCD{1}|{sub3}~|}  & \texttt{\{sub3\}}  \\
			$x_4$      & {\large \textLCD{1}|{sub4}~|}  & \texttt{\{sub4\}}  \\
			$x_5$      & {\large \textLCD{1}|{sub5}~|}  & \texttt{\{sub5\}}  \\
			$x_6$      & {\large \textLCD{1}|{sub6}~|}  & \texttt{\{sub6\}}  \\
			$\bar x$   & {\large \textLCD{1}|{barx}~|}  & \texttt{\{barx\}}  \\
			$\bar y$   & {\large \textLCD{1}|{bary}~|}  & \texttt{\{bary\}}  \\ \hline
		\end{tabular}

		\begin{tabular}{|l|l|l|}
			\multicolumn{3}{c}{\textbf{Redefined characters}}                \\ \hline
			Name       & Symbol                         & \verb|\LCD| Code   \\ \hline
			!          & {\large \textLCD{1}|{!}~|}     & \texttt{\{!\}}     \\
			\{         & {\large \textLCD{1}|{lb}~|}    & \texttt{\{lb\}}    \\
			\}         & {\large \textLCD{1}|{rb}~|}    & \texttt{\{rb\}}    \\
			$[$        & {\large \textLCD{1}|{rb}~|}    & \texttt{\{rb\}}    \\
			$]$        & {\large \textLCD{1}|{rb}~|}    & \texttt{\{rb\}}    \\
			\textsurd  & {\large \textLCD{1}|{sqrt}~|}  & \texttt{\{sqrt\}}  \\
			e          & {\large \textLCD{1}|{e}~|}     & \texttt{e}         \\
			i          & {\large \textLCD{1}|{i}~|}     & \texttt{i}         \\
			v          & {\large \textLCD{1}|{v}~|}     & \texttt{v}         \\
			w          & {\large \textLCD{1}|{w}~|}     & \texttt{w}         \\ \hline
		\end{tabular}
	\end{multicols}
\end{center}

\part{Buttons}

\section{Usage}

The \verb|\tibtn...| commands only prints the button and are useful when
displaying buttons inline. The \verb|\tibtnextra...| commands shows the extra
functionality of the button (accessed by the \tibtnsecond{} and \tibtnalpha{}
buttons). See secion \ref{sec:button} for defined buttons and section
\ref{sec:buttonscustom} for how to define custom buttons.

\section{Defined buttons} \label{sec:button}

\begin{center}
	\begin{longtable}{lrlr}
		\verb|\tibtnsecond|     & \tibtnsecond     & \tibtnextrasecond     & \verb|\tibtnextrasecond|     \\
		\verb|\tibtnmode|       & \tibtnmode       & \tibtnextramode       & \verb|\tibtnextramode|       \\
		\verb|\tibtndel|        & \tibtndel        & \tibtnextradel        & \verb|\tibtnextradel|        \\
		\verb|\tibtnalpha|      & \tibtnalpha      & \tibtnextraalpha      & \verb|\tibtnextraalpha|      \\
		\verb|\tibtnxton|       & \tibtnxton       & \tibtnextraxton       & \verb|\tibtnextraxton|       \\
		\verb|\tibtnstat|       & \tibtnstat       & \tibtnextrastat       & \verb|\tibtnextrastat|       \\
		\verb|\tibtnmath|       & \tibtnmath       & \tibtnextramath       & \verb|\tibtnextramath|       \\
		\verb|\tibtnmatrix|     & \tibtnmatrix     & \tibtnextramatrix     & \verb|\tibtnextramatrix|     \\
		\verb|\tibtnprgm|       & \tibtnprgm       & \tibtnextraprgm       & \verb|\tibtnextraprgm|       \\
		\verb|\tibtnvars|       & \tibtnvars       & \tibtnextravars       & \verb|\tibtnextravars|       \\
		\verb|\tibtnclear|      & \tibtnclear      & \tibtnextraclear      & \verb|\tibtnextraclear|      \\
		\verb|\tibtnxnone|      & \tibtnxnone      & \tibtnextraxnone      & \verb|\tibtnextraxnone|      \\
		\verb|\tibtnsin|        & \tibtnsin        & \tibtnextrasin        & \verb|\tibtnextrasin|        \\
		\verb|\tibtncos|        & \tibtncos        & \tibtnextracos        & \verb|\tibtnextracos|        \\
		\verb|\tibtntan|        & \tibtntan        & \tibtnextratan        & \verb|\tibtnextratan|        \\
		\verb|\tibtnpower|      & \tibtnpower      & \tibtnextrapower      & \verb|\tibtnextrapower|      \\
		\verb|\tibtnxtwo|       & \tibtnxtwo       & \tibtnextraxtwo       & \verb|\tibtnextraxtwo|       \\
		\verb|\tibtncomma|      & \tibtncomma      & \tibtnextracomma      & \verb|\tibtnextracomma|      \\
		\verb|\tibtnleftparen|  & \tibtnleftparen  & \tibtnextraleftparen  & \verb|\tibtnextraleftparen|  \\
		\verb|\tibtnrightparen| & \tibtnrightparen & \tibtnextrarightparen & \verb|\tibtnextrarightparen| \\
		\verb|\tibtndiv|        & \tibtndiv        & \tibtnextradiv        & \verb|\tibtnextradiv|        \\
		\verb|\tibtnlog|        & \tibtnlog        & \tibtnextralog        & \verb|\tibtnextralog|        \\
		\verb|\tibtnseven|      & \tibtnseven      & \tibtnextraseven      & \verb|\tibtnextraseven|      \\
		\verb|\tibtneight|      & \tibtneight      & \tibtnextraeight      & \verb|\tibtnextraeight|      \\
		\verb|\tibtnnine|       & \tibtnnine       & \tibtnextranine       & \verb|\tibtnextranine|       \\
		\verb|\tibtntimes|      & \tibtntimes      & \tibtnextratimes      & \verb|\tibtnextratimes|      \\
		\verb|\tibtnln|         & \tibtnln         & \tibtnextraln         & \verb|\tibtnextraln|         \\
		\verb|\tibtnfour|       & \tibtnfour       & \tibtnextrafour       & \verb|\tibtnextrafour|       \\
		\verb|\tibtnfive|       & \tibtnfive       & \tibtnextrafive       & \verb|\tibtnextrafive|       \\
		\verb|\tibtnsix|        & \tibtnsix        & \tibtnextrasix        & \verb|\tibtnextrasix|        \\
		\verb|\tibtnminus|      & \tibtnminus      & \tibtnextraminus      & \verb|\tibtnextraminus|      \\
		\verb|\tibtnsto|        & \tibtnsto        & \tibtnextrasto        & \verb|\tibtnextrasto|        \\
		\verb|\tibtnone|        & \tibtnone        & \tibtnextraone        & \verb|\tibtnextraone|        \\
		\verb|\tibtntwo|        & \tibtntwo        & \tibtnextratwo        & \verb|\tibtnextratwo|        \\
		\verb|\tibtnthree|      & \tibtnthree      & \tibtnextrathree      & \verb|\tibtnextrathree|      \\
		\verb|\tibtnplus|       & \tibtnplus       & \tibtnextraplus       & \verb|\tibtnextraplus|       \\
		\verb|\tibtnon|         & \tibtnon         & \tibtnextraon         & \verb|\tibtnextraon|         \\
		\verb|\tibtnzero|       & \tibtnzero       & \tibtnextrazero       & \verb|\tibtnextrazero|       \\
		\verb|\tibtndot|        & \tibtndot        & \tibtnextradot        & \verb|\tibtnextradot|        \\
		\verb|\tibtnneg|        & \tibtnneg        & \tibtnextraneg        & \verb|\tibtnextraneg|        \\
		\verb|\tibtnenter|      & \tibtnenter      & \tibtnextraenter      & \verb|\tibtnextraenter|      \\
	\end{longtable}
\end{center}

\section{Custom buttons} \label{sec:buttonscustom}

Buttons are defined using the \verb|\tibtn| and \verb|\tibtnextra|. It's often
convenient to define a \verb|\tibtn| command then use it inside
\verb|\tibtnextra| (See example below). See section \ref{sec:colordef} for a
list of defined colors.

\verb|\tibtn[<text color>]{<button color>}{<text>}|

\verb|\tibtnextra{<middle>}{<top left>}{<top right>}|

\begin{SideBySideExample}[xrightmargin=5.5cm]
\def\tibtnfoo{\tibtn[black]{tibtnblue}{FOO}}
\tibtnfoo{}

\def\tibtnextrafoo{\tibtnextra{\tibtnfoo}{BAR}{BAZ}}
\tibtnextrafoo
\end{SideBySideExample}

\end{document}
