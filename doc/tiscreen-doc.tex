\documentclass[a4paper,12pt]{article}
\usepackage[iso,english]{isodate}
\usepackage[margin=1in]{geometry}
\usepackage[english]{babel}
\usepackage{parskip}
\usepackage[color]{tiscreen}
\usepackage{multicol}
\usepackage{hyperref}
\usepackage{xspace}
\usepackage{array}

\usepackage{listings}
\lstset{
	numbers=left, numberstyle=\tiny,
	frame=single,
	basicstyle=\footnotesize\ttfamily,
	language=Tex,
}

\newcommand{\LCDsymb}[1]{\large \textLCD{1}|{#1}~|}
\newcommand{\LCDcode}[1]{\texttt{\{#1\}}}
\newcommand{\LCDcmd}{\textbackslash\texttt{LCD}\xspace}

\title{TI calculator display\\{\small(TI-82 STATS)}}
\author{Mustafa Ibrahim}

\begin{document}

\maketitle
\begin{center}
\tiscreen
|10^6*(4^7+1){rarrow}X  |
|       1.6385E10|
|{sqrt}(X)*X/10!      |
|     577971782.1|
|cos{ar}(cos(Ans))  |
|        62.11246|
|{fcur}               |
|                |
\end{center}
\tableofcontents
\newpage

\section{Quickstart}
\tiscreen
|4+1             |
|               5|
|Ans{sq}         |
|              25|
|                |
|                |
|                |
|                |

\begin{lstlisting}
\documentclass{article}
\usepackage[color]{tiscreen}
% Remove 'color' to display in back and white

\begin{document}

\tiscreen
|4+1             |
|               5|
|Ans{sq}         |
|              25|
|                |
|                |
|                |
|                |

\end{document}
\end{lstlisting}

\section{Package option(s)}
\subsection{Color}
Using the \texttt{color} option will change the colors used by the \LCDcmd
command. The colors are defined as \texttt{tiscreenfg} (foreground. i.e. font
color) and \texttt{tiscreenbg} (background). These colors can be redefined like
this:

\begin{lstlisting}
% Add this to your preamble
\definecolor{tiscreenbg}{HTML}{5d9345}
\definecolor{tiscreenfg}{HTML}{FFFFFF}
\end{lstlisting}

\section{LCD size}
The default LCD size is $8\times 16$ (the size of the TI-82
STATS). It can be changed by redefining the variables used to
determine the size of the display or by using the original
\LCDcmd command.

\begin{lstlisting}
% First method:
\def\tiscreenX{16}
\def\tiscreenY{8}

% Second method:
\LCD{5}{11}
|ANOTHER  |
|EXAMPLE  |
|WITH A   |
|DIFFERENT|
|SIZE     |
\end{lstlisting}

\section{Additional defined characters} \LCDcolors{black}{white}
\begin{tabular}{l|l|>{\ttfamily}l}
	\hline
	\multicolumn{3}{c}{\textbf{Added characters}} \\
	\hline \hline
	Name		& Symbol		&\normalfont{\LCDcmd Code} \\
	\hline
	E		& \LCDsymb{sciE}	& \{sciE\} \\
	$\sigma$	& \LCDsymb{sigma}	& \{sigma\} \\
	$\bar x$	& \LCDsymb{barx}	& \{barx\} \\
	$\bar y$	& \LCDsymb{bary}	& \{bary\} \\
	$x^2$		& \LCDsymb{sq}		& \{sq\} \\
	$x^{-1}$	& \LCDsymb{ar}		& \{ar\} \\
	$x_1$		& \LCDsymb{sub1}	& \{sub1\} \\
	$x_2$		& \LCDsymb{sub2}	& \{sub2\} \\
	$x_3$		& \LCDsymb{sub3}	& \{sub3\} \\
	$x_4$		& \LCDsymb{sub4}	& \{sub4\} \\
	$x_5$		& \LCDsymb{sub5}	& \{sub5\} \\
	$x_6$		& \LCDsymb{sub6}	& \{sub6\} \\
	\hline
	\multicolumn{3}{c}{\textbf{Redefined characters}} \\
	\hline \hline
	!		& \LCDsymb{!}		& \{!\} \\
	$\sqrt{x}$	& \LCDsymb{sqrt}	& \{sqrt\} \\
	e		& \LCDsymb{e}		& e \\
	i		& \LCDsymb{i}		& i \\
	v		& \LCDsymb{v}		& v \\
	w		& \LCDsymb{w}		& w \\
\end{tabular}

\end{document}
