\documentclass[a4paper,12pt]{article}
\usepackage[margin=1in]{geometry}
\usepackage[english]{babel}
\usepackage{parskip}
\usepackage[color]{tiscreen}
\usepackage{multicol}

\usepackage{listings}
\lstset{
	numbers=left, numberstyle=\tiny,
	frame=single,
	basicstyle=\footnotesize\ttfamily,
	language=Tex,
}

\newcommand{\LCDsymb}[1]{\large \textLCD{1}|{#1}~|}
\newcommand{\LCDcode}[1]{\texttt{\{#1\}}}

\title{TI calculator display\\{\small(TI-82 STATS)}}
\author{Mustafa Ibrahim}
\date{2021/05/17}

\begin{document}

\maketitle
\begin{center}
\tiscreen
|10^6*(4^7+1){rarrow}X  |
|       1.6385E10|
|{sqrt}(X)*X/10!      |
|     577971782.1|
|cos{ar}(cos(Ans))  |
|        62.11246|
|{fcur}               |
|                |
\end{center}
\tableofcontents
\newpage

\section{Quickstart}
\tiscreen
|4+1             |
|               5|
|Ans{sq}            |
|              25|
|                |
|                |
|                |
|                |

\begin{lstlisting}
\documentclass{article}
\usepackage[color]{tiscreen}
% Remove 'color' to display in back and white

\begin{document}

\tiscreen
|4+1             |
|               5|
|Ans{sq}            |
|              25|
|                |
|                |
|                |
|                |

\end{document}
\end{lstlisting}

\section{Package option(s)}
\subsection{Color}
Using the \texttt{color} option will change the colors used by the
\texttt{\textbackslash LCD} command. The colors are defined as
\texttt{tiscreenfg} (foreground. i.e. font color) and \texttt{tiscreenbg}
(background). These colors can be redefined like this:

\begin{lstlisting}
% Add this to your preamble
\definecolor{tiscreenbg}{HTML}{5d9345}
\definecolor{tiscreenfg}{HTML}{FFFFFF}
\end{lstlisting}

\section{LCD size}
The default LCD size is $8\times 16$ (the size of the TI-82 STATS). It can be
changed by redefining the variables used to determine the size of the display
or by using the original {\textbackslash LCD} command.

\begin{lstlisting}
% First method:
\def\tiscreenX{16}
\def\tiscreenY{8}

% Second method:
\LCD{5}{11}
|ANOTHER  |
|EXAMPLE  |
|WITH A   |
|DIFFERENT|
|SIZE     |
\end{lstlisting}

\section{Additional defined characters} \LCDcolors{black}{white}
\begin{tabular}{l|l|l}
	Name & Symbol & Code \\
	\hline
	E (scientific notation) & \LCDsymb{sciE} & \LCDcode{sciE} \\
	Sigma (lowercase) & \LCDsymb{sigma} & \LCDcode{sigma} \\
	$\bar x$ & \LCDsymb{barx} & \LCDcode{barx} \\
	$\bar y$ & \LCDsymb{bary} & \LCDcode{bary} \\
	$^\wedge 2$ (square root) & \LCDsymb{sq} & \LCDcode{sq} \\
	$^\wedge (-1)$ & \LCDsymb{ar} & \LCDcode{ar} \\
\end{tabular}

\subsection{Redefined characters}
Predefined characters that where redefined to match the TI-82 STATS.

\begin{tabular}{l|l|l}
	Name & Symbol & Code \\
	\hline
	e & \LCDsymb{e} & \texttt{e} \\
	i & \LCDsymb{i} & \texttt{i} \\
	Square root & \LCDsymb{sqrt} & \LCDcode{sqrt} \\
	! & \LCDsymb{!} & \LCDcode{!} \\
\end{tabular}

\end{document}
