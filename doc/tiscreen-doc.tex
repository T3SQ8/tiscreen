% TODO: make funciton for printing \LCDcmd

\documentclass[a4paper,12pt]{article}
\usepackage[iso,english]{isodate}
\usepackage[margin=1in]{geometry}
\usepackage[english]{babel}
\usepackage{parskip}
\usepackage[color]{tiscreen}
\usepackage{multicol}
\usepackage{hyperref}
\usepackage{array}
\usepackage{fancyvrb-ex}
\usepackage{amsmath}
\fvset{
	frame=single,
	label=\fbox{Source},
	framesep=4mm,
	numbers=left,
	fontsize=\footnotesize,
}

\newcommand{\LCDsymb}[1]{\large \textLCD{1}|{#1}~|}
\newcommand{\printcmd}[1]{\texttt{\textbackslash #1}}

\title{TI calculator screen (and buttons)\\{\small TI-82 STATS, TI-84}}
\author{Mustafa Ibrahim, Caleb Bibb}

\begin{document}

\maketitle

\begin{center}
\tiscreen
|10^6*(4^7+1){rarrow}X  |
|       1.6385E10|
|{sqrt}(X)*X/10!      |
|     577971782.1|
|cos{ar}(cos(Ans))  |
|        62.11246|
|{fcur}               |
|                |

\tibtnmatrixextra
\end{center}

\tableofcontents\newpage

\section{Quickstart}

\begin{SideBySideExample}[xrightmargin=5.5cm]
%\usepackage[color]{tiscreen}

\tiscreen
|4+1             |
|               5|
|Ans{sq}         |
|              25|
|{fcur}          |
|                |
|                |
|                |

\tibtnmatrixextra

Lorem
\tibtn[white]{tibtngray}{MATH}
ipsum

\tibtnextra{\tibtn[white]{tibtngray}{MATH}}{TEST}{A}
\end{SideBySideExample}

\section{Package option(s)}
\subsection{Color}

Using the \texttt{color} option will change the colors used by the
\printcmd{LCD} command for printing the screen using \printcmd{tiscreen}. The
colors are defined as \texttt{tiscreenfg} (foreground, i.e. font color) and
\texttt{tiscreenbg} (background) and redefined like this:

\begin{Verbatim}
% Add this to your preamble
\definecolor{tiscreenbg}{HTML}{5d9345}
\definecolor{tiscreenfg}{HTML}{FFFFFF}
\end{Verbatim}

\newpage

\part{Screen}
\section{LCD size}

The default LCD size is $8 \times 16$ (the size of the TI-82 STATS). It can be
changed by redefining the variables used to determine the size of the display
or by using the original \printcmd{LCD} command.

\begin{Verbatim}
% First method (For entire document)
\def\tiscreenX{16}
\def\tiscreenY{8}

% Second method (Only once)
\LCD{5}{11}
|ANOTHER  |
|EXAMPLE  |
|WITH A   |
|DIFFERENT|
|SIZE     |
\end{Verbatim}

\section{Additional defined characters} \LCDcolors{black}{white}
\begin{center}
	\begin{tabular}{|l|l|>{\ttfamily}l|}
		\multicolumn{3}{c}{\textbf{Added characters}} \\
		\hline
		Name       & Symbol          &\textnormal{\printcmd{LCD} Code} \\
		\hline
		E          & \LCDsymb{sciE}  & \{sciE\} \\
		$\sigma$   & \LCDsymb{sigma} & \{sigma\} \\
        $\Sigma$   & \LCDsymb{Sigma} & \{Sigma\} \\
		$x^2$      & \LCDsymb{sq}    & \{sq\} \\
		$x^{-1}$   & \LCDsymb{ar}    & \{ar\} \\
        $x^3$      & \LCDsymb{c3}    & \{c3\} \\
		$y^x$      & \LCDsymb{cx}    & \{cx\} \\
		$x_1$      & \LCDsymb{sub1}  & \{sub1\} \\
		$x_2$      & \LCDsymb{sub2}  & \{sub2\} \\
		$x_3$      & \LCDsymb{sub3}  & \{sub3\} \\
		$x_4$      & \LCDsymb{sub4}  & \{sub4\} \\
		$x_5$      & \LCDsymb{sub5}  & \{sub5\} \\
		$x_6$      & \LCDsymb{sub6}  & \{sub6\} \\
        $x_{10}$   & \LCDsymb{sub10} & \{sub10\} \\
		$\bar x$   & \LCDsymb{barx}  & \{barx\} \\
		$\bar y$   & \LCDsymb{bary}  & \{bary\} \\
        $-x$       & \LCDsymb{dash}  & \{dash\} \\
        $\rightarrow$ & \LCDsymb{sto}& \{sto\} \\
        $\theta$   & \LCDsymb{theta} & \{theta\} \\
        $\pi$      & \LCDsymb{pi}    & \{pi\} \\
        $\eta$     & \LCDsymb{eta}   & \{eta\} \\
        $+$        & \LCDsymb{tick}   & \{tick\} \\
        $^\circ$   & \LCDsymb{degree}& \{degree\} \\
                   & \LCDsymb{square}& \{square\} \\
                   & \LCDsymb{alpha} & \{alpha\} \\
        $\uparrow$ & \LCDsymb{2nd}   & \{2nd\} \\
        $\geq$     & \LCDsymb{geq}   & \{geq\} \\
        $\leq$     & \LCDsymb{leq}   & \{leq\} \\
        $\neq$     & \LCDsymb{neq}   & \{neq\} \\
        $x^3$      & \LCDsymb{c3}    & \{c3\} \\
        $A^T$      & \LCDsymb{transpose}& \{transpose\} \\
        $A^r$      & \LCDsymb{upr}   & \{upr\} \\
        $chi$      & \LCDsymb{chi}   & \{chi\} \\
        $\triangleright$ & \LCDsymb{fwedge}& \{fwedge\} \\
        $\triangleleft$ & \LCDsymb{bwedge} & \{bwedge\} \\
        $\mathbf{N}$ & \LCDsymb{bbN} & \{bbN\} \\
        $\mathbf{/}$ & \LCDsymb{bb/} & \{bb/\} \\
		\hline
    \end{tabular}
    \newpage
    \begin{tabular}{|l|l|>{\ttfamily}l|}
		\multicolumn{3}{c}{\textbf{Redefined characters}} \\
		\hline
		!          & \LCDsymb{!}     & \{!\} \\
		\{         & \LCDsymb{lb}    & \{lb\} \\
		\}         & \LCDsymb{rb}    & \{rb\} \\
		$[$        & \LCDsymb{[}     & \{[\} \\
		$]$        & \LCDsymb{]}     & \{]\} \\
		$\sqrt{x}$ & \LCDsymb{sqrt}  & \{sqrt\} \\
		e          & \LCDsymb{e}     & \{e\} \\
		i          & \LCDsymb{i}     & \{i\} \\
		v          & \LCDsymb{v}     & \{v\} \\
		w          & \LCDsymb{w}     & \{w\} \\
        Ellipses   & \LCDsymb{ell}   & \{ell\} \\
        Apostrophe & \LCDsymb{'}     & \{'\} \\
        List       & \LCDsymb{L}     & \{L\} \\
        Underscore & \LCDsymb{_}     & \{\_\} \\
		\hline
	\end{tabular}
\end{center}

\part{Buttons}
\section{Usage}

Use the \printcmd{tibtn} command only prints the button and is usefull when
needing the buttons to be displayed inline. The \printcmd{tibtnextra} takes
extra arguments to show extra options for the button (accesed by the
\tibtnsecond{} and \tibtnalpha{} buttons).

\begin{SideBySideExample}[xrightmargin=5.5cm]
\tibtnextra{\tibtn[white]{tibtngray}{MATH}}{TEST}{A}

Lorem
\tibtn[white]{tibtngray}{MATH}
ipsum
\end{SideBySideExample}

\section{Defined buttons}

\begin{center}
	\begin{tabular}{|>{\ttfamily}l|l|}
		\multicolumn{2}{c}{\textbf{Defined buttons}}      \\ \hline
		\textnormal{Command}        & Output              \\ \hline
		\printcmd{tibtnmatrix}      & \tibtnmatrix{}      \\ \hline
		\printcmd{tibtnmatrixextra} & \tibtnmatrixextra{} \\ \hline
	\end{tabular}
\end{center}

\end{document}